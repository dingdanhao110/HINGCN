%\documentclass[sigconf]{acmart}

%\documentclass[sigconf, anonymous, review]{acmart}
\documentclass[sigconf,anonymous]{acmart}

\usepackage{booktabs} % For formal tables

\usepackage{enumitem}
\usepackage{multirow}
\usepackage{hhline}
\usepackage{pifont}
\usepackage{amsmath}
\usepackage{bm}
\usepackage{array}
\usepackage{algorithm}
\usepackage{algpseudocode}
\usepackage{xcolor}
\usepackage{wasysym}
\usepackage{hyperref}
\usepackage{subfig}

\renewcommand{\algorithmicrequire}{\textbf{Input:}}  % Use Input in the format of Algorithm
\renewcommand{\algorithmicensure}{\textbf{Output:}} % Use Output in the format of Algorithm
\algnewcommand{\LeftComment}[1]{\State \(\triangleright\) #1}

%\newdef{definition}{Definition}
\newtheorem{theorem}{Theorem}
%\newtheorem{lemma}{Lemma}
\newtheorem{proposition}{Proposition}
\newtheorem{summary}{Summary}
\newtheorem{example}{Example}
\newtheorem{conclusion}{Conclusion}

\newcommand{\sij}{S_{ij}}
\newcommand{\zij}{Z_{ij}}
\newcommand{\bme}{{\bm e}}
\newcommand{\bmv}{{\bm v}}
\newcommand{\bmz}{{\bm z}}
\newcommand{\bmw}{{\bm w}}
\newcommand{\ben}[1]{\textcolor{red}{#1}}
\newcommand{\ev}{eigenvector}
\newcommand{\pev}{pseudo-eigenvector}
\newcommand{\xarrow}[2]{x_{#1} \rightarrow x_{#2}}

\newcommand{\ngd}{\emph{20ngD}} 
\newcommand{\mnist}{\emph{MNIST0127}}
\newcommand{\yale}{\emph{Yale\_5\textsc{class}}}
\newcommand{\coil}{\emph{COIL20}}
\newcommand{\isolet}{\emph{isolet\_5\textsc{class}}}
\newcommand{\seg}{\emph{seg\_7\textsc{class}}}
\newcommand{\yeast}{\emph{Yeast\_4\textsc{class}}}
\newcommand{\glass}{\emph{glass}}


\long\def\comment#1{}

\setcopyright{rightsretained}
%\setcopyright{usgov}
%\setcopyright{usgovmixed}
%\setcopyright{cagov}
%\setcopyright{cagovmixed}

\settopmatter{printacmref=false, printfolios=false}
\pagestyle{plain}

\comment{
% DOI
\acmDOI{10.475/123_4}

% ISBN
\acmISBN{123-4567-24-567/08/06}

%Conference
\acmConference[WOODSTOCK'97]{ACM Woodstock conference}{July 1997}{El
  Paso, Texas USA} 
\acmYear{1997}
\copyrightyear{2016}


\acmArticle{4}
\acmPrice{15.00}
}

% These commands are optional
%\acmBooktitle{Transactions of the ACM Woodstock conference}
%\editor{Jennifer B. Sartor}
%\editor{Theo D'Hondt}
%\editor{Wolfgang De Meuter}


\begin{document}
\title{ROSC: Robust Spectral Clustering on Multi-scale Data}

%\numberofauthors{1} %  in this sample file, there are a *total*
% of EIGHT authors. SIX appear on the 'first-page' (for formatting
% reasons) and the remaining two appear in the \additionalauthors section.

\author{Ben Trovato}
\authornote{Dr.~Trovato insisted his name be first.}
\orcid{1234-5678-9012}
\affiliation{%
  \institution{Institute for Clarity in Documentation}
  \streetaddress{P.O. Box 1212}
  \city{Dublin} 
  \state{Ohio} 
  \postcode{43017-6221}
}
\email{trovato@corporation.com}

\author{G.K.M. Tobin}
\authornote{The secretary disavows any knowledge of this author's actions.}
\affiliation{%
  \institution{Institute for Clarity in Documentation}
  \streetaddress{P.O. Box 1212}
  \city{Dublin} 
  \state{Ohio} 
  \postcode{43017-6221}
}
\email{webmaster@marysville-ohio.com}

\author{Lars Th{\o}rv{\"a}ld}
\authornote{This author is the
  one who did all the really hard work.}
\affiliation{%
  \institution{The Th{\o}rv{\"a}ld Group}
  \streetaddress{1 Th{\o}rv{\"a}ld Circle}
  \city{Hekla} 
  \country{Iceland}}
\email{larst@affiliation.org}

\author{Valerie B\'eranger}
\affiliation{%
  \institution{Inria Paris-Rocquencourt}
  \city{Rocquencourt}
  \country{France}
}
\author{Aparna Patel} 
\affiliation{%
 \institution{Rajiv Gandhi University}
 \streetaddress{Rono-Hills}
 \city{Doimukh} 
 \state{Arunachal Pradesh}
 \country{India}}
\author{Huifen Chan}
\affiliation{%
  \institution{Tsinghua University}
  \streetaddress{30 Shuangqing Rd}
  \city{Haidian Qu} 
  \state{Beijing Shi}
  \country{China}
}

\author{Charles Palmer}
\affiliation{%
  \institution{Palmer Research Laboratories}
  \streetaddress{8600 Datapoint Drive}
  \city{San Antonio}
  \state{Texas} 
  \postcode{78229}}
\email{cpalmer@prl.com}

\author{John Smith}
\affiliation{\institution{The Th{\o}rv{\"a}ld Group}}
\email{jsmith@affiliation.org}

\author{Julius P.~Kumquat}
\affiliation{\institution{The Kumquat Consortium}}
\email{jpkumquat@consortium.net}

% The default list of authors is too long for headers.
\renewcommand{\shortauthors}{X. Li et al.}

\newcommand{\bmx}{\bm{x}}

\comment{
\begin{abstract}
We investigate the effectiveness of spectral methods in clustering multi-scale data, which is data whose
clusters are of various sizes and densities. 
We review existing spectral methods that are designed to handle multi-scale data
and propose an alternative approach that is orthogonal to existing methods. 
We put forward the algorithm ROSC, which computes an affinity matrix $\tilde{Z}$ that takes 
into account both objects' feature similarity and reachability similarity. 
We perform extensive experiments comparing ROSC against 9 other spectral methods
on both real and synthetic datasets. 
%Our results show that ROSC significantly outperforms others over a wide spectrum of datasets.  
Our results show that ROSC performs very well against the competitors. 
In particular, it is very robust in that it consistently performs well over all the datasets tested.
Also, it outperforms others by wide margins for datasets that are highly multi-scale.
\end{abstract}
}
%\begin{CCSXML}
%<ccs2012>
% <concept>
%  <concept_id>10010520.10010553.10010562</concept_id>
%  <concept_desc>Computer systems organization~Embedded systems</concept_desc>
%  <concept_significance>500</concept_significance>
% </concept>
% <concept>
%  <concept_id>10010520.10010575.10010755</concept_id>
%  <concept_desc>Computer systems organization~Redundancy</concept_desc>
%  <concept_significance>300</concept_significance>
% </concept>
% <concept>
%  <concept_id>10010520.10010553.10010554</concept_id>
%  <concept_desc>Computer systems organization~Robotics</concept_desc>
%  <concept_significance>100</concept_significance>
% </concept>
% <concept>
%  <concept_id>10003033.10003083.10003095</concept_id>
%  <concept_desc>Networks~Network reliability</concept_desc>
%  <concept_significance>100</concept_significance>
% </concept>
%</ccs2012>  
%\end{CCSXML}
%
%\ccsdesc[500]{Computer systems organization~Embedded systems}
%\ccsdesc[300]{Computer systems organization~Redundancy}
%\ccsdesc{Computer systems organization~Robotics}
%\ccsdesc[100]{Networks~Network reliability}
%

%\keywords{Clustering, robust spectral clustering, multi-scale data}

\maketitle


\comment{
\begin{definition}
\label{def:grouping}
\textbf{(Grouping effect)}. 
Given a set of objects $\mathcal{X} = \{x_1, x_2,..., x_n\}$,
let $\bmw_q$ be the $q$-th column of $\mathcal{W}$. 
Further, let $\xarrow{i}{j}$ denote the condition:
(1) $\bmx_i^T \bmx_j \rightarrow 1$ and 
(2) $\lVert \bmw_i - \bmw_j \rVert_2 \rightarrow 0$.
A Matrix $Z$ is said to have grouping effect
if
\[
\xarrow{i}{j} \Rightarrow |Z_{ip} - Z_{jp}| \rightarrow 0\; \forall 1 \leq p \leq n.
\]
\end{definition}
}

This technical report proves the grouping effect of $|Z^*|$,$|(Z^*)^T|$ and $\tilde{Z}$.
In the following discussion, we use $\bmz_q^*$ to denote the $q$-th column vector of $Z^*$.

\begin{lemma}
\label{lemma1}
Given a set of objects $\mathcal{X}$,
the matrix
$X\in \mathcal{R}^{p\times n}$ that is composed of the {\pev}s as rows,
 the reachability matrix $\mathcal{W}$,
 and the optimal soution $Z^*$,
\begin{equation}
\label{eq:zi}
Z_{ip}^* = \frac{\bm{x}_i^T(\bm{x}_p-X\bm{z}_p^*) + \alpha_2 \mathcal{W}_{ip}}{\alpha_1+\alpha_2}, \;\;\; \forall 1 \leq i, p \leq n.
\end{equation}
\end{lemma}

\begin{proof}
For $1 \leq p \leq n$,
let $J(\bm{z}_p) =  ||\bm{x}_p-X\bm{z}_p||_2^2 + \alpha_1 ||\bm{z}_p||_2^2 + \alpha_2 ||\bm{z}_p-\bm{w}_p||_2^2$.
Since $Z^*$ is the optimal solution, we have $\frac{\partial{J}}{\partial{Z}_{ip}}|_{\bm{z}_p = \bm{z}_p^*} = 0\; \forall 1\leq i \leq n$.
Hence, $-2\bm{x}_i^T(\bm{x}_p-X\bm{z}_p^*)+2\alpha_1Z_{ip}^*+2\alpha_2(Z_{ip}^*-\mathcal{W}_{ip}) = 0$,
which induces Equation~\ref{eq:zi}.
\end{proof}

\begin{lemma}
\label{lemma2}
%Given a set of objects $X\in R^{d\times n}$ and a TKNN graph $\mathcal{G} = (\mathcal{V}, \mathcal{E}, \mathcal{W}_K)$,
%let $\bm{x}_p$ be an arbitrary object, 
%$\bm{w}_p$ be the column vector in $\mathcal{W}_K$ describing the connectivity of $\bm{x}_p$, and
%$\bm{z}_p^*$ be the optimal solution to the problem 
%$\min_{\bm{z}_p} ||\bm{x}_p-X\bm{z}_p||_2^2 + \lambda_1 ||\bm{z}_p||_2^2 + \lambda_2 ||\bm{z}_p-\bm{w}_p||_2^2$.
%Assume all the objects have been normalized.
%For any two objects $\bm{x}_i$ and $\bm{x}_j$,
$\forall 1 \leq i, j, p \leq n$,
\begin{equation}
\label{eq:norm}
|Z_{ip}^*-Z_{jp}^*| \leq \frac{c\sqrt{2(1-r)} + \alpha_2|\mathcal{W}_{ip}-\mathcal{W}_{jp}|}{\alpha_1+\alpha_2},
\end{equation}
where $c = \sqrt{1+\alpha_2||\bm{w}_p||_2^2}$ and $r = \bm{x}_i^T\bm{x}_j$.
\end{lemma}

\begin{proof}
From Equation~\ref{eq:zi}, we have 
\begin{equation}
\nonumber
Z_{ip}^*-Z_{jp}^* = \frac{(\bm{x}_i^T - \bm{x}_j^T)(\bm{x}_p-X\bm{z}_p^*) + \alpha_2 (\mathcal{W}_{ip}-\mathcal{W}_{jp})}{\alpha_1+\alpha_2}.
\end{equation}
That implies 
\begin{small}
\begin{equation}
\label{eq:zizj}
\begin{split}
|Z_{ip}^*-Z_{jp}^*| & \leq \frac{|(\bm{x}_i^T - \bm{x}_j^T)(\bm{x}_p-X\bm{z}_p^*)| + \alpha_2 |(\mathcal{W}_{ip}-\mathcal{W}_{jp})|}{\alpha_1+\alpha_2}\\
& \leq \frac{||(\bm{x}_i^T - \bm{x}_j^T)||_2||(\bm{x}_p-X\bm{z}_p^*)||_2 + \alpha_2 |(\mathcal{W}_{ip}-\mathcal{W}_{jp})|}{\alpha_1+\alpha_2}\\
\end{split}
\end{equation}
\end{small}

Since the column vectors of $X$ are normalized (i.e., $\bmx_q^T \bmx_q = 1 \; \forall 1 \leq q \leq n$) , we have
$||(\bm{x}_i^T - \bm{x}_j^T)||_2 = \sqrt{2(1-r)}$,
where $r = \bm{x}_i^T\bm{x}_j$.
%measuring the closeness between $\bm{x}_i$ and $\bm{x}_j$ in the feature space.
%As $\bm{z}_p^*$ is the optimal solution, 
As $Z^*$ is the optimal solution, we have
\begin{equation}
\begin{split}
J(\bm{z}_p^*) & = ||\bm{x}_p-X\bm{z}_p^*||_2^2 + \alpha_1 ||\bm{z}_p^*||_2^2 + \alpha_2 ||\bm{z}_p^*-\bm{w}_p||_2^2 \leq  \\
J(\bm{0}) & = ||\bm{x}_p||_2^2 + \alpha_2 ||\bm{w}_p||_2^2 = 1 + \alpha_2 ||\bm{w}_p||_2^2.
\end{split}
\end{equation}
Hence, $||\bm{x}_p-X\bm{z}_p^*||_2 \leq \sqrt{1 + \alpha_2 ||\bm{w}_p||_2^2} = c$.
Equation~\ref{eq:zizj} can be further simplified as
\begin{equation}
\nonumber
|Z_{ip}^*-Z_{jp}^*| \leq \frac{c\sqrt{2(1-r)}+ \alpha_2 |(\mathcal{W}_{ip}-\mathcal{W}_{jp})|}{\alpha_1+\alpha_2}.
\end{equation}
\end{proof}

\begin{lemma}
$Z^*$ and $|Z^*|$ have grouping effect.
\label{lemma:z-star}
\end{lemma}
\begin{proof}
Given two objects $x_i$ and $x_j$ such that $\xarrow{i}{j}$,
we have,  %by definition, 
(1) $\bmx_i^T \bmx_j \rightarrow 1$ and (2) $||\bmw_{i}-\bmw_{j}||_2 \rightarrow 0$.
These imply
$r = \bmx_i^T \bmx_j \rightarrow 1$ and  $|\mathcal{W}_{ip}-\mathcal{W}_{jp}| \rightarrow 0$.
Hence, the two terms of the numerator of the R.H.S of Equation~\ref{eq:norm} are close to 0. 
Therefore, $|Z_{ip}^*-Z_{jp}^*| \rightarrow 0$ and thus $Z^*$ has grouping effect.
Further,
we have 
$\bigl||Z_{ip}^*|-|Z_{jp}^*|\bigr| \leq |Z_{ip}^*-Z_{jp}^*|$,
so $\bigl||Z_{ip}^*|-|Z_{jp}^*|\bigr| \rightarrow 0$ and $|Z^*|$ has grouping effect.
\end{proof}

\begin{lemma}
\label{lemma3}
$(Z^*)^T$ and $|(Z^*)^T|$ have grouping effect. 
\begin{proof}
The problem is equivalent to if $x_i \rightarrow x_j$, $Z^*_{pi} \rightarrow Z^*_{pj}$.
From Lemma~\ref{lemma1},
\begin{equation}
\nonumber
z_{pi}^*-z_{pj}^* = \frac{\bm{x}_p^T ( \bmx_i - \bmx_j - X(\bm{z}_i^*-\bm{z}_j^*)) + \alpha_2 (W_{pi}-W_{pj})}{\alpha_1+\alpha_2}.
\end{equation}
Since $\bm{z}_i^* = (X^TX + \alpha_1I + \alpha_2I)^{-1}(X^T\bmx_i+\alpha_2\bm{w}_i),\bm{z}_j^* = (X^TX + \alpha_1I + \alpha_2I)^{-1}(X^T\bmx_j+\alpha_2\bm{w}_j)$,
let $Y = X(X^TX + \alpha_1I + \alpha_2I)^{-1}$. Then
\begin{equation}
\begin{split}
|Z_{pi}^*-Z_{pj}^*| & \leq \frac{||\bmx_p^T(\bm{x}_i - \bm{x}_j)||_2 + ||\bmx_p^TX(\bm{z}_i^*-\bm{z}_j^*)||_2 + \alpha_2 |(W_{pi}-W_{pj})|}{\alpha_1+\alpha_2}\\
& \leq \frac{||\bmx_p^T(\bm{x}_i - \bm{x}_j)||_2 + ||\bmx_p^TYX^T(\bmx_i-\bmx_j)||_2}{\alpha_1+\alpha_2}\\
& + \frac{\alpha_2||\bmx_p^TY(\bm{w}_i - \bm{w}_j)||_2 + \alpha_2 |W_{pi}-W_{pj}|}{\alpha_1+\alpha_2}\\
\end{split}
\end{equation}
If $x_i \rightarrow x_j$, i.e., $\bmx_i^T\bmx_j \rightarrow 1$ and $||\bm{w}_i - \bm{w}_j||_2 \rightarrow 0$, 
we have
$||\bm{x}_i - \bm{x}_j||_2 \rightarrow 0$ and $|W_{pi}-W_{pj}| \rightarrow 0$.
Then $|Z_{pi}^*-Z_{pj}^*| \rightarrow 0$ and
$(Z^*)^T$ has grouping effect.
Since $\bigl||Z_{pi}^*|-|Z_{pj}^*|\bigr| \leq |Z_{pi}^*-Z_{pj}^*|$, $|(Z^*)^T|$ also has grouping effect.
\end{proof}
\end{lemma}

\begin{lemma}
\label{lemma4}
Matrix $\tilde{Z}$ has grouping effect. 
\begin{proof}
From Lemma~\ref{lemma:z-star} and~\ref{lemma3},
both $\lvert Z^*\rvert$ and $\lvert (Z^*)^T\rvert$ have the grouping effect.
Since $\tilde{Z} = (|Z^*|+|(Z^*)^T|)/2$,
\begin{equation}
\begin{split}
\lvert \tilde{Z}_{ip} - \tilde{Z}_{jp}\rvert & = \frac{\bigl| (|Z^*_{ip}|+|Z^*_{pi}|) - (|Z^*_{jp}|+|Z^*_{pj}|)\bigr|}{2} \\
& \leq \frac{|Z_{ip}^*-Z_{jp}^*| + |Z_{pi}^*-Z_{pj}^*|}{2}
\end{split}
\end{equation}
If $x_i \rightarrow x_j$,
both $|Z_{ip}^*-Z_{jp}^*| \rightarrow 0$ and $|Z_{pi}^*-Z_{pj}^*| \rightarrow 0$,
so $\lvert \tilde{Z}_{ip} - \tilde{Z}_{jp}\rvert \rightarrow 0$ and
$\tilde{Z}$ has grouping effect.
\end{proof}
\end{lemma}





\newpage



%\end{document}  % This is where a 'short' article might terminate

% ensure same length columns on last page (might need two sub-sequent latex runs)

%ACKNOWLEDGMENTS are optional
\comment{
\section{Acknowledgments}
This section is optional; it is a location for you
to acknowledge grants, funding, editing assistance and
what have you.  In the present case, for example, the
authors would like to thank Gerald Murray of ACM for
his help in codifying this \textit{Author's Guide}
and the \textbf{.cls} and \textbf{.tex} files that it describes.
}

% The following two commands are all you need in the
% initial runs of your .tex file to
% produce the bibliography for the citations in your paper.
\bibliographystyle{ACM-Reference-Format}

%\bibliographystyle{abbrv}

\bibliography{sc}  % vldb_sample.bib is the name of the Bibliography in this case
% You must have a proper ".bib" file
%  and remember to run:
% latex bibtex latex latex
% to resolve all references

%APPENDIX is optional.
% ****************** APPENDIX **************************************
% Example of an appendix; typically would start on a new page
%pagebreak


\end{document}
