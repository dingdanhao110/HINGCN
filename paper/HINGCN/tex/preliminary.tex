\section{Preliminary}
\label{sec:pre}

\begin{definition}
\textbf{ Heterogeneous graphs}. 
\dan{TODO}
\hfill$\Box$
\end{definition}

\begin{example}
\dan{TODO}
\end{example}

\begin{definition}
\textbf{Metapath}\citep{SunHYYW11}. 
A meta-path $\Phi$ is a path defined on a heterogeneous graph. Meta-path $\Phi$: $T_1 \stackrel{R_1}{\longrightarrow} \cdots \stackrel{R_l}{\longrightarrow} T_{l+1}$ defines a composite relation $R = R_1 \circ \cdots \circ R_l$ that relates objects of type $T_1$ to objects of type $T_{l+1}$. If two objects $x_u$ and $x_v$ are related by the composite relation $R$, then there is a path, denoted by $p_{x_u \rightsquigarrow x_v}$, that connects $x_u$ to $x_v$ in $G$. Moreover, in this paper we say $x_v$ is a meta-path $\Phi$ related neighbor of $x_u$, denoted by $x_v \in N^\Phi(x_u)$

\end{definition}

\begin{example}
In Fig.~\ref{figure:example1},
\dan{TODO}
\end{example}

The notations throughout the rest of paper are shown in Table \ref{table:notation}.


\begin{table}[]
\caption{Description of symbols}
\centering
\begin{tabular}{cc}
\hline
Notation       & Description                                                       \\ \hline
$\Phi$         & Meta-path                                                         \\
$N^\Phi$       & neighbor with respect to a given meta-path $\Phi$          \\
$h^{(t)}$        & Node embedding at layer $t$; $h^0$ is input node feature \\
$r^{(t)}$        & Edge embedding at layer $t$; $r^0$ is input edge feature \\
$e_{i,j}^\Phi$ & Similarity of node pair ($i$,$j$) under meta-path $\Phi$   \\
$a_{i,j}^\Phi$ & Attention of node pair ($i$,$j$) under meta-path $\Phi$    \\
$W_F$          & Weight matrix of fully connected layer $F$          \\
$\sigma$       & Sigmoid function                                                  \\
$z_i$            & Output embedding for node $i$                                     \\
\hline
\end{tabular}
\label{table:notation}
\end{table}











