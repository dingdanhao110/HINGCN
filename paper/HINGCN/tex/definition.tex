\section{Definitions}
\label{sec:definition}
In this section we give a formal problem definition.
%Table~\ref{table:symbol} lists some of the symbols used in this paper.

\comment{
\begin{table}
\caption{Descriptions of symbols}
\centering
\scriptsize
%\scriptsize
\begin{tabular}{|c|l|}
    \hline
    {\bf Notation} & \multicolumn{1}{c|}{\bf Description} \\ \hline
    $G = (V, E)$ 	& An HIN $G$ with object set $V$ and link set $E$ \\ \hline
    $\mathcal{T}, T_i, m$	&$\mathcal{T} = \{T_1,T_2,...,T_m\}$, a set of $m$ object types\\ \hline
    $\mathcal{A}_i, \mathcal{A}$	&Attribute set $\mathcal{A}_i$ of object type $T_i$, $\mathcal{A} = \bigcup_{i=1}^m{\mathcal{A}_i}$\\ \hline
    $G_{\mathcal{A}} = (V, E, \mathcal{A})$ 	& An attributed HIN $G_{\mathcal{A}}$ with attribute set $\mathcal{A}$ \\ \hline
%    $x, e, T$	&An arbitrary object $x$, edge $e$, object type $T$\\ \hline
%    $x_{ij}$					&  The $j$-th type-$T_i$ object\\ \hline
    $\mathcal{X}_i, n_i$	& The set $\mathcal{X}_i$ of type $T_i$ objects, $|\mathcal{X}_i| = n_i$ \\ \hline
    $T_G = (\mathcal{T}, \mathcal{R})$		& Network schema of HIN $G$ \\ \hline
%    $n, n_i$ 				&   $|\mathcal{V}|, |\mathcal{X}_i|$ \\ \hline
    $p_{x_u\leadsto{x_v}} \vdash \mathcal{P}$ & $p_{x_u\leadsto{x_v}}$ is an instance of the meta-path $\mathcal{P}$ \\ \hline
%    $\mathcal{P}, \mathcal{PS}$ 	& Meta path $\mathcal{P}$, a set of meta paths $\mathcal{PS}$\\ \hline
    $G_{T_i,\mathcal{P}}$, $G_\mathcal{P}$			                 & TSSN of object type $T_i$ induced by meta-path $\mathcal{P}$\\ \hline
%    $\mathcal{L}$, $L(x)$					& Training set $\mathcal{L}$; $L(x)$ = label of object $x$ if $x \in \mathcal{L}$ \\ \hline
    $F_i$				                                            	& Type $T_i$ object attribute matrix $\in {\mathbb{R}^{n_i\times {|\mathcal{A}_i|}}}$ \\ \hline
    $\mathcal{M}$, $\mathcal{C}$					& Must-link set and cannot-link set \\ \hline
    $\mathfrak{C} = \{C_1, ..., C_k\}$			                 	& A set of $k$ disjoint clusters  \\ \hline
   % $\Upsilon_{C}$, $\Upsilon_{G_\mathcal{P}}$, $\Upsilon_G$	& Cohesiveness of a cluster, a sub-network and an HIN\\ \hline
   % $\Psi_{C}$, $\Psi_{G_\mathcal{P}}$, $\Psi_G$	& Compactness of a cluster, a sub-network and an HIN\\ \hline
   
%    $\Upsilon_{C}$, $\Upsilon_{G_\mathcal{P}}$, $\Upsilon_G$	& Cohesiveness of a cluster, a sub-network and an HIN\\ \hline
%    $\Psi_{C}$, $\Psi_{G_\mathcal{P}}$, $\Psi_G$	& Compactness of a cluster, a sub-network and an HIN\\ \hline
%    $Q(\mathcal{P})$					& The proportion of qualified objects for meta path $\mathcal{P}$ \\ \hline
%    $E(\mathcal{P})$					& The effectiveness of meta path $\mathcal{P}$ \\ \hline
%    $U(\mathcal{P})$					& The usefulness of meta path $\mathcal{P}$ \\ \hline
%   
\end{tabular}
\label{table:symbol}
%\end{small}
\end{table}
}


\comment{
\begin{definition}
\textbf{Heterogeneous Information Network (HIN)}. Let $\cal{T}$ = $\{ t_1, \ldots, t_m \}$ be a set of $m$ object types.
Let $\cal{V}$ be a set of  $n$ objects.
Each object $x \in \cal{V}$ has a unique type, denoted by $T(x)$.
We use $\mathcal{X}_h$ to denote the set of all objects of type $t_h$,
i.e., $\mathcal{X}_h = \{x \in \mathcal{V} \; | \; T(x) = t_h \}$
and $\mathcal{V} = \cup_{h=1}^{m} \mathcal{X}_h$.
Let $n_h = |\mathcal{X}_h|$.
Let $\mathcal{E}$ be a set of links, which represents a binary relation between objects in $\mathcal{V}$.
An object $x_i$ is related to another object $x_j$ iff $(x_i, x_j) \in \mathcal{E}$.
An HIN $\mathcal{G} = \langle \mathcal{V}, \mathcal{E} \rangle$
is a set of objects $\mathcal{V}$ and a set of links $\mathcal{E}$.
\end{definition}
}

\begin{definition}
\label{def:ahin}
\textbf{Attributed Heterogeneous Information Network (AHIN)}. 
Let $\mathcal{T}$ = $\{T_1, ..., T_m \}$ be a set of $m$ object types.
For each type $T_i$, 
let $\mathcal{X}_i$ be the set of objects of type $T_i$
and $A_i$ be the set of attributes defined for objects of type $T_i$.
An object $x_j$ of type $T_i$ is associated with an attribute vector
$\bm f_{j} = (f_{j1}, f_{j2}, ..., f_{j|A_i|})$.
%For each type $T_i$, 
%let $n_i$ and $\mathcal{X}_i = \{ x_{i1}, ..., x_{in_i} \}$ be the number and the set of objects of type $T_i$, respectively.
An AHIN is a graph $G = (V, E, \mathcal{A})$, where
$V = \bigcup_{i=1}^m\mathcal{X}_i$ is a set of nodes,
$E$ is a set of links, each represents a binary relation between two objects in $V$,
and $\mathcal{A} = \bigcup_{i=1}^m{A_i}$.
If $m$ = 1 (i.e., there is only one object type), $G$ reduces to a homogeneous information network.
\hfill$\Box$
\end{definition}

%Given an HIN $G$, objects of type $T$ are usually attached with a set of \textbf{attributes} $A$
%and these attributes can be divided into different types.
%For example, restaurants in Yelp may have numerical attributes like the number of reviews, longitude and latitude, 
%ordinal attribute star level, boolean attribute reservation, etc.
%These attributes convey different information and may be useful in different clustering tasks.
%In this paper, we consider all these kinds of attributes.
%Including attributes to an HIN $G = (V, E)$, an \textbf{attributed heterogeneous information network (AHIN)} $G_{\mathcal{A}} = (V, E, \mathcal{A})$,
%where $\mathcal{A} = \bigcup_{i=1}^m{A_i}$.
%For each object $x_i$ of type $T$, the attribute vector is represented as $\bm f_{i} = (f_{i1}, f_{i2}, ..., f_{i|A|})$.

\begin{definition}
\textbf{Network schema}. A network schema is the meta template of
an AHIN $G = (V, E, \mathcal{A})$.
Let 
(1) $\phi: V \rightarrow \mathcal{T}$ be an object-type mapping that maps an object in $V$ into its type, and
(2) $\psi: E\rightarrow \mathcal{R}$ be a link-relation mapping that maps a link in $E$ into a relation in a set of 
relations $\mathcal{R}$.
The network schema of an AHIN $G$, denoted by $T_G = (\mathcal{T}, \mathcal{R})$,
shows how objects of different types are related by the relations in $\mathcal{R}$. 
$T_G$ can be represented by a {\it schematic graph} with $\mathcal{T}$ and $\mathcal{R}$ being the node set and the edge set, 
respectively.
Specifically, there is an edge ($T_i$, $T_j$) in the schematic graph iff there is a relation in $\mathcal{R}$ that relates
objects of type $T_i$ to objects of type $T_j$.
\hfill$\Box$
%
%
% with the object type mapping $\phi: V\rightarrow \mathcal{T}$ and the link mapping $\psi: E\rightarrow \mathcal{R}$, which is a directed graph defined over object types $\mathcal{T}$, with edges as relations from $\mathcal{R}$, denoted as $T_G = (\mathcal{T}, \mathcal{R})$.
\end{definition}

Figure~\ref{figure:subnetworks}(a) shows an example AHIN that models movie information 
(attribute information is not shown).
The AHIN consists of four object types: $\mathcal{T}$ = \{
movie ($\Diamond$), actor($\Box$), director($\Circle$), producer($\triangle$) \}.
There are also three relations in $\mathcal{R}$, which are illustrated by the three edges in the schematic
graph (Figure~\ref{figure:subnetworks}(b)). For example, the relation between {\it actor} and {\it movie} 
carries the information of which actor has acted in which movie.
Actors, directors and producers have attributes like age, gender, birthplace, while 
movies are associated with attributes like release date, box office, etc. 


\comment{
\begin{definition}
\textbf{Meta path}. A meta-path $\mathcal{P} = t_{i_1} \ldots t_{i_d}$ is a sequence of object types.
We use $\mathcal{P}[j]$ to denote the $j$-th object type ($t_{i_j}$) specified in the meta-path $\mathcal{P}$.
Given an HIN $G$ and a path $p$ in $G$,
let $|p|$ be the length (as measured by the number of objects) of path $p$ and $p[j]$ be the $j$-th object in $p$.
Path $p$
is an \emph{instance} of $\mathcal{P}$, denoted by $p \vdash \mathcal{P}$ iff
$p$ and $\mathcal{P}$ are of the same length and the types of the objects in $p$ match
those specified in $\mathcal{P}$. Formally,
$p \vdash \mathcal{P}$ iff
(1) $|p|$ = $d$ and
(2) $T(p[j]) = \mathcal{P}[j] \; \forall 1 \leq j \leq d$.
\end{definition}
}

\begin{definition}
\textbf{Meta-path}. A meta-path $\mathcal{P}$ is a path defined on the 
schematic graph of a network schema.
%$T_G = (\mathcal{T}, \mathcal{R})$.
A meta-path $\mathcal{P}$: $T_1\stackrel{R_1}{\longrightarrow} \cdots\stackrel{R_l}{\longrightarrow} T_{l+1}$
 defines a composite relation $R = R_1\circ \cdots \circ R_l$ 
 that relates objects of type $T_1$ to objects of type $T_{l+1}$.
 If two objects $x_u$ and $x_v$ are related by the composite relation $R$,
 then there is a path, denoted by $p_{x_u\leadsto{x_v}}$, that connects $x_u$ to $x_v$ in $G$.
 Moreover, the sequence of links in $p_{x_u\leadsto{x_v}}$ matches the sequence of relations
 $R_1$, ..., $R_l$ based on the link-relation mapping $\psi$.
 We say that $p_{x_u\leadsto{x_v}}$ is a {\it path instance} of $\mathcal{P}$, denoted by
 $p_{x_u\leadsto{x_v}} \vdash \mathcal{P}$.
 \hfill$\Box$
%  where $\circ$ denotes the composition operator on relations.
\end{definition}

As an example, the path $p_{M1 \leadsto M3} = M1\rightarrow A2 \rightarrow M3$
in Figure~\ref{figure:subnetworks}(a)
is an instance of  the meta-path Movie-Actor-Movie (abbrev. MAM). 
%
%
% has been applied to various data mining tasks including transductive classification, and it has been proved useful. Grempt utilizes meta paths where $T_1 = T_{l+1} $, and converts the original heterogeneous information network to several homogeneous \emph{topology shrinking sub-networks}.

%\begin{small}
\begin{figure}
    \centering
    \vspace{-3mm}
        \includegraphics[width = 0.9\linewidth]{figure/hin.pdf}
        \caption{{\scriptsize An AHIN (a) and its schematic graph (b)}}
        \label{figure:subnetworks}
        \vspace{-2mm}
\end{figure}
%\end{small}

\comment{
\begin{definition}
\textbf{Topology Shrinking Sub-network (TSSN)}. Given an HIN $G = (V, E)$, 
the TSSN of a certain object type $T_i$ derived from a meta-path $\mathcal{P}$
is a graph
whose nodes consist of only objects of type $T_i$ and whose edges connect objects that are related by
instances of $\mathcal{P}$. Formally,
the TSSN is the graph
 $G_{T_i,\mathcal{P}} = (\mathcal{X}_i, E_{T_i})$,
where 
$E_{T_i} = \{e_{uv} | p_{x_u\leadsto{x_v}} \vdash {\mathcal{P}}, x_u, x_v \in{\mathcal{X}_i}\}$.
We write $G_\mathcal{P}$ instead of $G_{T_i,\mathcal{P}}$ if the object type $T_i$ is implicitly known.
\hfill$\Box$
%
% can be denoted as $G_T = (V_T, E_T, R_T)$, $V_T = \mathcal{X}_T$, $E_T = \{e_{uv}|p_{x_u\leadsto{x_v}}\in{\mathcal{P}}, x_u, x_v \in{V_T}\}$ and $R_T = \{R_{uv}|R_{uv}$ is the weight of $e_{uv}\in{E_T}\}$, where $p_{x_u\leadsto{x_v}}$ denotes a path instance between $x_u$ and $x_v$.
\end{definition}

Figures \ref{figure:subnetworks}(c)-(e) show the TSSNs of type Movie derived from the meta-paths
MAM, MDM and MPM, respectively.
A TSSN shows how objects of a certain type are related by the composite relation given by a meta-path. 
For example, the meta-path MAM relates two movie objects if those movies share an actor. 
In Figure~\ref{figure:subnetworks}(c), movies M1 and M2 are connected by an edge in the TSSN because
actor A1 acted in both movies.
}

\begin{definition}
\label{def:user-guidance}
\textbf{Supervision constraint}.
The clustering process is supervised by a user through a constraint ($\mathcal{M}$, $\mathcal{C}$), where $\mathcal{M}$
and $\mathcal{C}$ are the {\it must-link set} and the {\it cannot-link set}, respectively. 
Each of these sets is a set of object pairs $(x_a, x_b)$.
An object pair in $\mathcal{M}$ represents that the two objects must belong to the same cluster,
while a pair in $\mathcal{C}$
indicates that the two objects  should not be put into the same cluster.
\hfill$\Box$
\end{definition}

\begin{definition}
\textbf{Semi-supervised clustering in an AHIN}.
\label{def:clustering}
Given an AHIN $G = (V,E,\mathcal{A})$, a supervision constraint ($\mathcal{M}$, $\mathcal{C}$),
a target object type $T_i$, the number of clusters $k$, and a set of meta-paths $\mathcal{PS}$,
the problem of semi-supervised clustering of type $T_i$ objects in $G$ is to
(1) discover an object similarity measure $S$ that is based on object attributes and meta-paths, and
(2) partition the objects in $\mathcal{X}_i$ into $k$ disjoint clusters $\mathfrak{C} = \{C_1, ..., C_k\}$
based on the similarity measure $S$ such that the clustering results best agree with the constraint 
($\mathcal{M}$, $\mathcal{C}$).
%\begin{itemize}\setlength{\itemsep}{-2pt}
%\item The object type $T$ to be clustered.
%\item The number of clusters $k$.
%\item A must-link set $\mathcal{M}$ and a cannot-link set $\mathcal{C}$.
%\item A set of meta paths $\mathcal{PS}$. 
%\end{itemize}
%and produces the following outputs:
%\begin{itemize}\setlength{\itemsep}{-2pt}
%\item A partitioning of the set $\mathcal{X}$  into $k$ disjoint clusters $\mathfrak{C} = \{C_1, ..., C_k\}$, where $\bigcup_{i=1}^k C_i = \mathcal{X}$ and
%$\bigcap_{i=1}^k C_i = \emptyset$.
%\item Meta path weight vector $\bm \lambda$, where $\lambda_i \geq 0$, $\sum_{i=1}^{|\mathcal{PS}|} \lambda_i = 1$.
%\item Attribute weight vector $\bm \omega$, where $\omega_i \geq 0$, $\sum_{i=1}^{|A|} \omega_i = 1$.
%\end{itemize}
\hfill$\Box$
\end{definition}


%\begin{example}
%Figure \ref{figure:subnetworks} shows topology shrinking sub-networks derived from an HIN by different meta paths. The network in the left is an movie related HIN composed of movies (M $\Diamond$), actors(A $\Box$), directors(D $\Circle$) and producers(P $\triangle$). By meta paths MAM, MDM and MPM, we can respectively derive homogeneous subnetworks (a), (b) and (c) for movies.
%\end{example}
%

%In this paper, we first investigate factors influencing the performance of transductive classification in HINs. Then based on the factors, we devise a black-box tester which can estimate the performance of transductive classification on a given HIN and thereby provide suggestions for users in whether choose transductive classification methods for a specific classification task.
%\begin{definition}
%\textbf{Black-box tester}. Given an HIN $G = (V, E)$, a specific classification task on objects of type $T_i$ with a set of labels ${LS}_i$, a set of meta paths $\mathcal{PS}$ and a training set of labeled objects $\mathcal{L}$, our objective is to provide a black-box tester which can judge the performance of transductive classification on $G$ and thereby guide users in choosing transductive classification. If the tester outputs \texttt{Yes}, it indicates good performance and strong recommendation. If the output is \texttt{No}, it represents poor performance and bad choice. In other cases, the tester will output \texttt{No idea and have a try}.
%\end{definition}


















