\section{Conclusions}
In this paper we studied the effectiveness of spectral clustering methods in handling multi-scale data.
We discussed the traditional approaches of locally scaling the similarity matrix and power-iteration-based
techniques. We described the methods ZP and FUSE, which were previously proposed to cluster multi-scale
data. 
We pointed out that these existing approaches focus on measuring the correlations of objects via 
feature similarity. However, for data with various sizes and densities, objects that belong to the same
big cluster could be at substantial distances from each other. Feature similarity could fail in this case.
In view of this, we proposed ROSC, which computes an affinity matrix $\tilde{Z}$ that takes into account
both feature similarity and reachability similarity. In particular, $\tilde{Z}$ is obtained by regularizing 
a primitive affinity matrix with a TKNN graph. 
We mathematically proved that $\tilde{Z}$ has the desired grouping effect, which makes it a very 
effective replacement of the similarity matrix $S$ used in spectral clustering. 
We conducted extensive experiments comparing the performance of ROSC against 9 other methods.
Our results show that ROSC provides very good performances across all the datasets, both real and
synthetic, we tested. In particular, for cases where the datasets are highly multi-scale, ROSC 
substantially outperforms other competitors. 
ROSC is therefore a very robust solution for clustering multi-scale data.
\label{sec:conclusion}
